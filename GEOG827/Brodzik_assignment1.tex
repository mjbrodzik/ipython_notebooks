
% Default to the notebook output style

    


% Inherit from the specified cell style.




    
\documentclass{article}

    
    
    \usepackage{graphicx} % Used to insert images
    \usepackage{adjustbox} % Used to constrain images to a maximum size 
    \usepackage{color} % Allow colors to be defined
    \usepackage{enumerate} % Needed for markdown enumerations to work
    \usepackage{geometry} % Used to adjust the document margins
    \usepackage{amsmath} % Equations
    \usepackage{amssymb} % Equations
    \usepackage{eurosym} % defines \euro
    \usepackage[mathletters]{ucs} % Extended unicode (utf-8) support
    \usepackage[utf8x]{inputenc} % Allow utf-8 characters in the tex document
    \usepackage{fancyvrb} % verbatim replacement that allows latex
    \usepackage{grffile} % extends the file name processing of package graphics 
                         % to support a larger range 
    % The hyperref package gives us a pdf with properly built
    % internal navigation ('pdf bookmarks' for the table of contents,
    % internal cross-reference links, web links for URLs, etc.)
    \usepackage{hyperref}
    \usepackage{longtable} % longtable support required by pandoc >1.10
    \usepackage{booktabs}  % table support for pandoc > 1.12.2
    

    
    
    \definecolor{orange}{cmyk}{0,0.4,0.8,0.2}
    \definecolor{darkorange}{rgb}{.71,0.21,0.01}
    \definecolor{darkgreen}{rgb}{.12,.54,.11}
    \definecolor{myteal}{rgb}{.26, .44, .56}
    \definecolor{gray}{gray}{0.45}
    \definecolor{lightgray}{gray}{.95}
    \definecolor{mediumgray}{gray}{.8}
    \definecolor{inputbackground}{rgb}{.95, .95, .85}
    \definecolor{outputbackground}{rgb}{.95, .95, .95}
    \definecolor{traceback}{rgb}{1, .95, .95}
    % ansi colors
    \definecolor{red}{rgb}{.6,0,0}
    \definecolor{green}{rgb}{0,.65,0}
    \definecolor{brown}{rgb}{0.6,0.6,0}
    \definecolor{blue}{rgb}{0,.145,.698}
    \definecolor{purple}{rgb}{.698,.145,.698}
    \definecolor{cyan}{rgb}{0,.698,.698}
    \definecolor{lightgray}{gray}{0.5}
    
    % bright ansi colors
    \definecolor{darkgray}{gray}{0.25}
    \definecolor{lightred}{rgb}{1.0,0.39,0.28}
    \definecolor{lightgreen}{rgb}{0.48,0.99,0.0}
    \definecolor{lightblue}{rgb}{0.53,0.81,0.92}
    \definecolor{lightpurple}{rgb}{0.87,0.63,0.87}
    \definecolor{lightcyan}{rgb}{0.5,1.0,0.83}
    
    % commands and environments needed by pandoc snippets
    % extracted from the output of `pandoc -s`
    \providecommand{\tightlist}{%
      \setlength{\itemsep}{0pt}\setlength{\parskip}{0pt}}
    \DefineVerbatimEnvironment{Highlighting}{Verbatim}{commandchars=\\\{\}}
    % Add ',fontsize=\small' for more characters per line
    \newenvironment{Shaded}{}{}
    \newcommand{\KeywordTok}[1]{\textcolor[rgb]{0.00,0.44,0.13}{\textbf{{#1}}}}
    \newcommand{\DataTypeTok}[1]{\textcolor[rgb]{0.56,0.13,0.00}{{#1}}}
    \newcommand{\DecValTok}[1]{\textcolor[rgb]{0.25,0.63,0.44}{{#1}}}
    \newcommand{\BaseNTok}[1]{\textcolor[rgb]{0.25,0.63,0.44}{{#1}}}
    \newcommand{\FloatTok}[1]{\textcolor[rgb]{0.25,0.63,0.44}{{#1}}}
    \newcommand{\CharTok}[1]{\textcolor[rgb]{0.25,0.44,0.63}{{#1}}}
    \newcommand{\StringTok}[1]{\textcolor[rgb]{0.25,0.44,0.63}{{#1}}}
    \newcommand{\CommentTok}[1]{\textcolor[rgb]{0.38,0.63,0.69}{\textit{{#1}}}}
    \newcommand{\OtherTok}[1]{\textcolor[rgb]{0.00,0.44,0.13}{{#1}}}
    \newcommand{\AlertTok}[1]{\textcolor[rgb]{1.00,0.00,0.00}{\textbf{{#1}}}}
    \newcommand{\FunctionTok}[1]{\textcolor[rgb]{0.02,0.16,0.49}{{#1}}}
    \newcommand{\RegionMarkerTok}[1]{{#1}}
    \newcommand{\ErrorTok}[1]{\textcolor[rgb]{1.00,0.00,0.00}{\textbf{{#1}}}}
    \newcommand{\NormalTok}[1]{{#1}}
    
    % Define a nice break command that doesn't care if a line doesn't already
    % exist.
    \def\br{\hspace*{\fill} \\* }
    % Math Jax compatability definitions
    \def\gt{>}
    \def\lt{<}
    % Document parameters
    \title{Brodzik\_assignment1}
    
    
    

    % Pygments definitions
    
\makeatletter
\def\PY@reset{\let\PY@it=\relax \let\PY@bf=\relax%
    \let\PY@ul=\relax \let\PY@tc=\relax%
    \let\PY@bc=\relax \let\PY@ff=\relax}
\def\PY@tok#1{\csname PY@tok@#1\endcsname}
\def\PY@toks#1+{\ifx\relax#1\empty\else%
    \PY@tok{#1}\expandafter\PY@toks\fi}
\def\PY@do#1{\PY@bc{\PY@tc{\PY@ul{%
    \PY@it{\PY@bf{\PY@ff{#1}}}}}}}
\def\PY#1#2{\PY@reset\PY@toks#1+\relax+\PY@do{#2}}

\expandafter\def\csname PY@tok@gd\endcsname{\def\PY@tc##1{\textcolor[rgb]{0.63,0.00,0.00}{##1}}}
\expandafter\def\csname PY@tok@gu\endcsname{\let\PY@bf=\textbf\def\PY@tc##1{\textcolor[rgb]{0.50,0.00,0.50}{##1}}}
\expandafter\def\csname PY@tok@gt\endcsname{\def\PY@tc##1{\textcolor[rgb]{0.00,0.27,0.87}{##1}}}
\expandafter\def\csname PY@tok@gs\endcsname{\let\PY@bf=\textbf}
\expandafter\def\csname PY@tok@gr\endcsname{\def\PY@tc##1{\textcolor[rgb]{1.00,0.00,0.00}{##1}}}
\expandafter\def\csname PY@tok@cm\endcsname{\let\PY@it=\textit\def\PY@tc##1{\textcolor[rgb]{0.25,0.50,0.50}{##1}}}
\expandafter\def\csname PY@tok@vg\endcsname{\def\PY@tc##1{\textcolor[rgb]{0.10,0.09,0.49}{##1}}}
\expandafter\def\csname PY@tok@m\endcsname{\def\PY@tc##1{\textcolor[rgb]{0.40,0.40,0.40}{##1}}}
\expandafter\def\csname PY@tok@mh\endcsname{\def\PY@tc##1{\textcolor[rgb]{0.40,0.40,0.40}{##1}}}
\expandafter\def\csname PY@tok@go\endcsname{\def\PY@tc##1{\textcolor[rgb]{0.53,0.53,0.53}{##1}}}
\expandafter\def\csname PY@tok@ge\endcsname{\let\PY@it=\textit}
\expandafter\def\csname PY@tok@vc\endcsname{\def\PY@tc##1{\textcolor[rgb]{0.10,0.09,0.49}{##1}}}
\expandafter\def\csname PY@tok@il\endcsname{\def\PY@tc##1{\textcolor[rgb]{0.40,0.40,0.40}{##1}}}
\expandafter\def\csname PY@tok@cs\endcsname{\let\PY@it=\textit\def\PY@tc##1{\textcolor[rgb]{0.25,0.50,0.50}{##1}}}
\expandafter\def\csname PY@tok@cp\endcsname{\def\PY@tc##1{\textcolor[rgb]{0.74,0.48,0.00}{##1}}}
\expandafter\def\csname PY@tok@gi\endcsname{\def\PY@tc##1{\textcolor[rgb]{0.00,0.63,0.00}{##1}}}
\expandafter\def\csname PY@tok@gh\endcsname{\let\PY@bf=\textbf\def\PY@tc##1{\textcolor[rgb]{0.00,0.00,0.50}{##1}}}
\expandafter\def\csname PY@tok@ni\endcsname{\let\PY@bf=\textbf\def\PY@tc##1{\textcolor[rgb]{0.60,0.60,0.60}{##1}}}
\expandafter\def\csname PY@tok@nl\endcsname{\def\PY@tc##1{\textcolor[rgb]{0.63,0.63,0.00}{##1}}}
\expandafter\def\csname PY@tok@nn\endcsname{\let\PY@bf=\textbf\def\PY@tc##1{\textcolor[rgb]{0.00,0.00,1.00}{##1}}}
\expandafter\def\csname PY@tok@no\endcsname{\def\PY@tc##1{\textcolor[rgb]{0.53,0.00,0.00}{##1}}}
\expandafter\def\csname PY@tok@na\endcsname{\def\PY@tc##1{\textcolor[rgb]{0.49,0.56,0.16}{##1}}}
\expandafter\def\csname PY@tok@nb\endcsname{\def\PY@tc##1{\textcolor[rgb]{0.00,0.50,0.00}{##1}}}
\expandafter\def\csname PY@tok@nc\endcsname{\let\PY@bf=\textbf\def\PY@tc##1{\textcolor[rgb]{0.00,0.00,1.00}{##1}}}
\expandafter\def\csname PY@tok@nd\endcsname{\def\PY@tc##1{\textcolor[rgb]{0.67,0.13,1.00}{##1}}}
\expandafter\def\csname PY@tok@ne\endcsname{\let\PY@bf=\textbf\def\PY@tc##1{\textcolor[rgb]{0.82,0.25,0.23}{##1}}}
\expandafter\def\csname PY@tok@nf\endcsname{\def\PY@tc##1{\textcolor[rgb]{0.00,0.00,1.00}{##1}}}
\expandafter\def\csname PY@tok@si\endcsname{\let\PY@bf=\textbf\def\PY@tc##1{\textcolor[rgb]{0.73,0.40,0.53}{##1}}}
\expandafter\def\csname PY@tok@s2\endcsname{\def\PY@tc##1{\textcolor[rgb]{0.73,0.13,0.13}{##1}}}
\expandafter\def\csname PY@tok@vi\endcsname{\def\PY@tc##1{\textcolor[rgb]{0.10,0.09,0.49}{##1}}}
\expandafter\def\csname PY@tok@nt\endcsname{\let\PY@bf=\textbf\def\PY@tc##1{\textcolor[rgb]{0.00,0.50,0.00}{##1}}}
\expandafter\def\csname PY@tok@nv\endcsname{\def\PY@tc##1{\textcolor[rgb]{0.10,0.09,0.49}{##1}}}
\expandafter\def\csname PY@tok@s1\endcsname{\def\PY@tc##1{\textcolor[rgb]{0.73,0.13,0.13}{##1}}}
\expandafter\def\csname PY@tok@kd\endcsname{\let\PY@bf=\textbf\def\PY@tc##1{\textcolor[rgb]{0.00,0.50,0.00}{##1}}}
\expandafter\def\csname PY@tok@sh\endcsname{\def\PY@tc##1{\textcolor[rgb]{0.73,0.13,0.13}{##1}}}
\expandafter\def\csname PY@tok@sc\endcsname{\def\PY@tc##1{\textcolor[rgb]{0.73,0.13,0.13}{##1}}}
\expandafter\def\csname PY@tok@sx\endcsname{\def\PY@tc##1{\textcolor[rgb]{0.00,0.50,0.00}{##1}}}
\expandafter\def\csname PY@tok@bp\endcsname{\def\PY@tc##1{\textcolor[rgb]{0.00,0.50,0.00}{##1}}}
\expandafter\def\csname PY@tok@c1\endcsname{\let\PY@it=\textit\def\PY@tc##1{\textcolor[rgb]{0.25,0.50,0.50}{##1}}}
\expandafter\def\csname PY@tok@kc\endcsname{\let\PY@bf=\textbf\def\PY@tc##1{\textcolor[rgb]{0.00,0.50,0.00}{##1}}}
\expandafter\def\csname PY@tok@c\endcsname{\let\PY@it=\textit\def\PY@tc##1{\textcolor[rgb]{0.25,0.50,0.50}{##1}}}
\expandafter\def\csname PY@tok@mf\endcsname{\def\PY@tc##1{\textcolor[rgb]{0.40,0.40,0.40}{##1}}}
\expandafter\def\csname PY@tok@err\endcsname{\def\PY@bc##1{\setlength{\fboxsep}{0pt}\fcolorbox[rgb]{1.00,0.00,0.00}{1,1,1}{\strut ##1}}}
\expandafter\def\csname PY@tok@mb\endcsname{\def\PY@tc##1{\textcolor[rgb]{0.40,0.40,0.40}{##1}}}
\expandafter\def\csname PY@tok@ss\endcsname{\def\PY@tc##1{\textcolor[rgb]{0.10,0.09,0.49}{##1}}}
\expandafter\def\csname PY@tok@sr\endcsname{\def\PY@tc##1{\textcolor[rgb]{0.73,0.40,0.53}{##1}}}
\expandafter\def\csname PY@tok@mo\endcsname{\def\PY@tc##1{\textcolor[rgb]{0.40,0.40,0.40}{##1}}}
\expandafter\def\csname PY@tok@kn\endcsname{\let\PY@bf=\textbf\def\PY@tc##1{\textcolor[rgb]{0.00,0.50,0.00}{##1}}}
\expandafter\def\csname PY@tok@mi\endcsname{\def\PY@tc##1{\textcolor[rgb]{0.40,0.40,0.40}{##1}}}
\expandafter\def\csname PY@tok@gp\endcsname{\let\PY@bf=\textbf\def\PY@tc##1{\textcolor[rgb]{0.00,0.00,0.50}{##1}}}
\expandafter\def\csname PY@tok@o\endcsname{\def\PY@tc##1{\textcolor[rgb]{0.40,0.40,0.40}{##1}}}
\expandafter\def\csname PY@tok@kr\endcsname{\let\PY@bf=\textbf\def\PY@tc##1{\textcolor[rgb]{0.00,0.50,0.00}{##1}}}
\expandafter\def\csname PY@tok@s\endcsname{\def\PY@tc##1{\textcolor[rgb]{0.73,0.13,0.13}{##1}}}
\expandafter\def\csname PY@tok@kp\endcsname{\def\PY@tc##1{\textcolor[rgb]{0.00,0.50,0.00}{##1}}}
\expandafter\def\csname PY@tok@w\endcsname{\def\PY@tc##1{\textcolor[rgb]{0.73,0.73,0.73}{##1}}}
\expandafter\def\csname PY@tok@kt\endcsname{\def\PY@tc##1{\textcolor[rgb]{0.69,0.00,0.25}{##1}}}
\expandafter\def\csname PY@tok@ow\endcsname{\let\PY@bf=\textbf\def\PY@tc##1{\textcolor[rgb]{0.67,0.13,1.00}{##1}}}
\expandafter\def\csname PY@tok@sb\endcsname{\def\PY@tc##1{\textcolor[rgb]{0.73,0.13,0.13}{##1}}}
\expandafter\def\csname PY@tok@k\endcsname{\let\PY@bf=\textbf\def\PY@tc##1{\textcolor[rgb]{0.00,0.50,0.00}{##1}}}
\expandafter\def\csname PY@tok@se\endcsname{\let\PY@bf=\textbf\def\PY@tc##1{\textcolor[rgb]{0.73,0.40,0.13}{##1}}}
\expandafter\def\csname PY@tok@sd\endcsname{\let\PY@it=\textit\def\PY@tc##1{\textcolor[rgb]{0.73,0.13,0.13}{##1}}}

\def\PYZbs{\char`\\}
\def\PYZus{\char`\_}
\def\PYZob{\char`\{}
\def\PYZcb{\char`\}}
\def\PYZca{\char`\^}
\def\PYZam{\char`\&}
\def\PYZlt{\char`\<}
\def\PYZgt{\char`\>}
\def\PYZsh{\char`\#}
\def\PYZpc{\char`\%}
\def\PYZdl{\char`\$}
\def\PYZhy{\char`\-}
\def\PYZsq{\char`\'}
\def\PYZdq{\char`\"}
\def\PYZti{\char`\~}
% for compatibility with earlier versions
\def\PYZat{@}
\def\PYZlb{[}
\def\PYZrb{]}
\makeatother


    % Exact colors from NB
    \definecolor{incolor}{rgb}{0.0, 0.0, 0.5}
    \definecolor{outcolor}{rgb}{0.545, 0.0, 0.0}



    
    % Prevent overflowing lines due to hard-to-break entities
    \sloppy 
    % Setup hyperref package
    \hypersetup{
      breaklinks=true,  % so long urls are correctly broken across lines
      colorlinks=true,
      urlcolor=blue,
      linkcolor=darkorange,
      citecolor=darkgreen,
      }
    % Slightly bigger margins than the latex defaults
    
    \geometry{verbose,tmargin=1in,bmargin=1in,lmargin=1in,rmargin=1in}
    
    

    \begin{document}
    
    
    \maketitle
    
    

    
    \subsubsection{GEOG827}\label{geog827}

\subsubsection{Assignment \#1}\label{assignment-1}

\subsubsection{M. J. Brodzik}\label{m.-j.-brodzik}

\subsubsection{due 3/12/17}\label{due-31217}

\subsection{Question 1}\label{question-1}

Using the tables and/or equations in the ``Calculating Evaporation''
documents posted in blackboard, notes and/or other sources (state the
source), express results as mean W/m2 over the day.

\subsection{Q1. Part 1}\label{q1.-part-1}

Part i) estimate the daily average solar radiation to the top of the
Earth's atmosphere at 56oN on July 2nd.

Total daily solar radiation to top of the atmosphere (TOA) is a function
of latitude, season and time of day. From Table D, ``Calculating
Evaporation Notes'', the Total daily solar radiation,
\(K_A\ [W m^{-2}]\), is given in 10-degree latitude increments, for Jun
22 and Jul 15. I begin by plotting these to see how much variability
there is in \(K_A\):

    \begin{Verbatim}[commandchars=\\\{\}]
{\color{incolor}In [{\color{incolor}1}]:} \PY{o}{\PYZpc{}}\PY{k}{pylab} inline
        \PY{k+kn}{import} \PY{n+nn}{matplotlib.pyplot} \PY{k+kn}{as} \PY{n+nn}{plt}
        \PY{k+kn}{import} \PY{n+nn}{numpy} \PY{k+kn}{as} \PY{n+nn}{np}
        
        \PY{c}{\PYZsh{} Define data to plot}
        \PY{n}{lats} \PY{o}{=} \PY{n}{np}\PY{o}{.}\PY{n}{arange}\PY{p}{(}\PY{l+m+mi}{10}\PY{p}{)} \PY{o}{*} \PY{l+m+mi}{10}
        \PY{n}{jun22} \PY{o}{=} \PY{n}{np}\PY{o}{.}\PY{n}{array}\PY{p}{(}\PY{p}{[}\PY{l+m+mf}{382.68}\PY{p}{,} \PY{l+m+mf}{422.88}\PY{p}{,} \PY{l+m+mf}{452.91}\PY{p}{,} \PY{l+m+mf}{472.29}\PY{p}{,} \PY{l+m+mf}{480.04}\PY{p}{,}
                          \PY{l+m+mf}{479.07}\PY{p}{,} \PY{l+m+mf}{474.23}\PY{p}{,} \PY{l+m+mf}{490.21}\PY{p}{,} \PY{l+m+mf}{513.46}\PY{p}{,} \PY{l+m+mf}{521.70}\PY{p}{]}\PY{p}{)}
        \PY{n}{jul15} \PY{o}{=} \PY{n}{np}\PY{o}{.}\PY{n}{array}\PY{p}{(}\PY{p}{[}\PY{l+m+mf}{387.52}\PY{p}{,} \PY{l+m+mf}{424.82}\PY{p}{,} \PY{l+m+mf}{450.49}\PY{p}{,} \PY{l+m+mf}{465.02}\PY{p}{,} \PY{l+m+mf}{468.41}\PY{p}{,}
                          \PY{l+m+mf}{462.12}\PY{p}{,} \PY{l+m+mf}{450.01}\PY{p}{,} \PY{l+m+mf}{452.43}\PY{p}{,} \PY{l+m+mf}{474.71}\PY{p}{,} \PY{l+m+mf}{481.49}\PY{p}{]}\PY{p}{)}
        
        \PY{c}{\PYZsh{} Make the plot:}
        \PY{n}{plt}\PY{o}{.}\PY{n}{rc}\PY{p}{(}\PY{l+s}{\PYZsq{}}\PY{l+s}{text}\PY{l+s}{\PYZsq{}}\PY{p}{,} \PY{n}{usetex}\PY{o}{=}\PY{n+nb+bp}{True}\PY{p}{)}
        \PY{n}{fig}\PY{p}{,} \PY{n}{ax} \PY{o}{=} \PY{n}{plt}\PY{o}{.}\PY{n}{subplots}\PY{p}{(}\PY{l+m+mi}{1}\PY{p}{)}
        \PY{n}{ax}\PY{o}{.}\PY{n}{plot}\PY{p}{(}\PY{n}{lats}\PY{p}{,} \PY{n}{jun22}\PY{p}{,} \PY{l+s}{\PYZsq{}}\PY{l+s}{co}\PY{l+s}{\PYZsq{}}\PY{p}{,} \PY{n}{label}\PY{o}{=}\PY{l+s}{\PYZsq{}}\PY{l+s}{jun22}\PY{l+s}{\PYZsq{}}\PY{p}{)}
        \PY{n}{ax}\PY{o}{.}\PY{n}{plot}\PY{p}{(}\PY{n}{lats}\PY{p}{,} \PY{n}{jul15}\PY{p}{,} \PY{l+s}{\PYZsq{}}\PY{l+s}{bo}\PY{l+s}{\PYZsq{}}\PY{p}{,} \PY{n}{label}\PY{o}{=}\PY{l+s}{\PYZsq{}}\PY{l+s}{jul15}\PY{l+s}{\PYZsq{}}\PY{p}{)}
        \PY{n}{ax}\PY{o}{.}\PY{n}{set\PYZus{}title}\PY{p}{(}\PY{l+s}{\PYZdq{}}\PY{l+s}{Total daily solar radiation to horizontal surface at TOA}\PY{l+s}{\PYZdq{}}\PY{p}{)}
        \PY{n}{ax}\PY{o}{.}\PY{n}{set\PYZus{}xlabel}\PY{p}{(}\PY{l+s}{\PYZdq{}}\PY{l+s}{Latitude}\PY{l+s}{\PYZdq{}}\PY{p}{)}
        \PY{n}{ax}\PY{o}{.}\PY{n}{set\PYZus{}ylabel}\PY{p}{(}\PY{l+s}{r\PYZsq{}}\PY{l+s}{\PYZdl{}K\PYZus{}A (W m\PYZca{}\PYZob{}\PYZhy{}2\PYZcb{})\PYZdl{}}\PY{l+s}{\PYZsq{}}\PY{p}{)}
        \PY{n}{ax}\PY{o}{.}\PY{n}{legend}\PY{p}{(}\PY{n}{loc}\PY{o}{=}\PY{l+s}{\PYZsq{}}\PY{l+s}{best}\PY{l+s}{\PYZsq{}}\PY{p}{,} \PY{n}{numpoints}\PY{o}{=}\PY{l+m+mi}{1}\PY{p}{)}
        \PY{n}{plt}\PY{o}{.}\PY{n}{show}\PY{p}{(}\PY{p}{)}
        \PY{n}{fig}\PY{o}{.}\PY{n}{savefig}\PY{p}{(}\PY{l+s}{\PYZsq{}}\PY{l+s}{HW1.1i\PYZus{}fig1.png}\PY{l+s}{\PYZsq{}}\PY{p}{)}
\end{Verbatim}

    \begin{Verbatim}[commandchars=\\\{\}]
Populating the interactive namespace from numpy and matplotlib
    \end{Verbatim}

    \begin{Verbatim}[commandchars=\\\{\}]
/Users/brodzik/.conda/envs/pmesdr/lib/python2.7/site-packages/matplotlib/font\_manager.py:273: UserWarning: Matplotlib is building the font cache using fc-list. This may take a moment.
  warnings.warn('Matplotlib is building the font cache using fc-list. This may take a moment.')
    \end{Verbatim}

    \begin{center}
    \adjustimage{max size={0.9\linewidth}{0.9\paperheight}}{Brodzik_assignment1_files/Brodzik_assignment1_1_2.png}
    \end{center}
    { \hspace*{\fill} \\}
    
    So although there is an inflection point in the data above 60N, for an
estimate I think it's sufficient to just linearly interpolate \(K_A\)
for the given dates to a value for Jul 2, and then interpolate to
56\(^{\circ}\) N.

    \begin{Verbatim}[commandchars=\\\{\}]
{\color{incolor}In [{\color{incolor}2}]:} \PY{c}{\PYZsh{} define a quick linear interpolation function}
        \PY{c}{\PYZsh{} calculate slope and intercept and the value of the line at }
        \PY{c}{\PYZsh{} the new value}
        \PY{k}{def} \PY{n+nf}{linear\PYZus{}model\PYZus{}value\PYZus{}at}\PY{p}{(}\PY{n}{x}\PY{p}{,} \PY{n}{x1}\PY{p}{,} \PY{n}{y1}\PY{p}{,} \PY{n}{x2}\PY{p}{,} \PY{n}{y2}\PY{p}{)}\PY{p}{:}
            \PY{n}{slope} \PY{o}{=} \PY{p}{(}\PY{n}{y2} \PY{o}{\PYZhy{}} \PY{n}{y1}\PY{p}{)} \PY{o}{/} \PY{p}{(}\PY{n}{x2} \PY{o}{\PYZhy{}} \PY{n}{x1}\PY{p}{)}
            \PY{c}{\PYZsh{} y = mx + b ==\PYZgt{} b = y \PYZhy{} mx}
            \PY{n}{intercept} \PY{o}{=}  \PY{n}{y1} \PY{o}{\PYZhy{}} \PY{p}{(}\PY{n}{slope} \PY{o}{*} \PY{n}{x1}\PY{p}{)}
            \PY{k}{return} \PY{p}{(}\PY{n}{slope} \PY{o}{*} \PY{n}{x}\PY{p}{)} \PY{o}{+} \PY{n}{intercept}
\end{Verbatim}

    \begin{Verbatim}[commandchars=\\\{\}]
{\color{incolor}In [{\color{incolor}3}]:} \PY{n}{KA\PYZus{}56N\PYZus{}jun22} \PY{o}{=} \PY{n}{linear\PYZus{}model\PYZus{}value\PYZus{}at}\PY{p}{(}\PY{l+m+mf}{56.}\PY{p}{,} \PY{l+m+mf}{50.}\PY{p}{,} \PY{l+m+mf}{479.07}\PY{p}{,} \PY{l+m+mf}{60.}\PY{p}{,} \PY{l+m+mf}{474.23}\PY{p}{)}
        \PY{n}{KA\PYZus{}56N\PYZus{}jul15} \PY{o}{=} \PY{n}{linear\PYZus{}model\PYZus{}value\PYZus{}at}\PY{p}{(}\PY{l+m+mf}{56.}\PY{p}{,} \PY{l+m+mf}{50.}\PY{p}{,} \PY{l+m+mf}{462.12}\PY{p}{,} \PY{l+m+mf}{60.}\PY{p}{,} \PY{l+m+mf}{450.01}\PY{p}{)}
\end{Verbatim}

    Linearly interpolate \(K_A\) at 56 N to July 2 between Jun22 and Jul15:

    \begin{Verbatim}[commandchars=\\\{\}]
{\color{incolor}In [{\color{incolor}4}]:} \PY{k+kn}{import} \PY{n+nn}{datetime}
        \PY{n}{jun22\PYZus{}doy} \PY{o}{=} \PY{n}{datetime}\PY{o}{.}\PY{n}{datetime}\PY{p}{(}\PY{l+m+mi}{2017}\PY{p}{,} \PY{l+m+mi}{6}\PY{p}{,} \PY{l+m+mi}{22}\PY{p}{)}\PY{o}{.}\PY{n}{timetuple}\PY{p}{(}\PY{p}{)}\PY{o}{.}\PY{n}{tm\PYZus{}yday}
        \PY{n}{jul2\PYZus{}doy} \PY{o}{=} \PY{n}{datetime}\PY{o}{.}\PY{n}{datetime}\PY{p}{(}\PY{l+m+mi}{2017}\PY{p}{,} \PY{l+m+mi}{7}\PY{p}{,} \PY{l+m+mi}{2}\PY{p}{)}\PY{o}{.}\PY{n}{timetuple}\PY{p}{(}\PY{p}{)}\PY{o}{.}\PY{n}{tm\PYZus{}yday}
        \PY{n}{jul15\PYZus{}doy} \PY{o}{=} \PY{n}{datetime}\PY{o}{.}\PY{n}{datetime}\PY{p}{(}\PY{l+m+mi}{2017}\PY{p}{,} \PY{l+m+mi}{7}\PY{p}{,} \PY{l+m+mi}{15}\PY{p}{)}\PY{o}{.}\PY{n}{timetuple}\PY{p}{(}\PY{p}{)}\PY{o}{.}\PY{n}{tm\PYZus{}yday}
        \PY{k}{print}\PY{p}{(}\PY{n}{jun22\PYZus{}doy}\PY{p}{,} \PY{n}{jul2\PYZus{}doy}\PY{p}{,} \PY{n}{jul15\PYZus{}doy}\PY{p}{)}
        
        \PY{n}{KA\PYZus{}56N\PYZus{}jul2} \PY{o}{=} \PY{n}{linear\PYZus{}model\PYZus{}value\PYZus{}at}\PY{p}{(}\PY{n}{jul2\PYZus{}doy}\PY{p}{,} 
                                            \PY{n}{jun22\PYZus{}doy}\PY{p}{,} \PY{n}{KA\PYZus{}56N\PYZus{}jun22}\PY{p}{,}
                                            \PY{n}{jul15\PYZus{}doy}\PY{p}{,} \PY{n}{KA\PYZus{}56N\PYZus{}jul15}\PY{p}{)}
        \PY{n}{KA\PYZus{}56N\PYZus{}jul2}
\end{Verbatim}

    \begin{Verbatim}[commandchars=\\\{\}]
(173, 183, 196)
    \end{Verbatim}

            \begin{Verbatim}[commandchars=\\\{\}]
{\color{outcolor}Out[{\color{outcolor}4}]:} 466.8999130434783
\end{Verbatim}
        
    Adding my interpolated values to the plot, I think it's sufficient to
use this approximation:

    \begin{Verbatim}[commandchars=\\\{\}]
{\color{incolor}In [{\color{incolor}5}]:} \PY{n}{fig}\PY{p}{,} \PY{n}{ax} \PY{o}{=} \PY{n}{plt}\PY{o}{.}\PY{n}{subplots}\PY{p}{(}\PY{l+m+mi}{1}\PY{p}{)}
        
        \PY{n}{ax}\PY{o}{.}\PY{n}{plot}\PY{p}{(}\PY{n}{lats}\PY{p}{,} \PY{n}{jun22}\PY{p}{,} \PY{l+s}{\PYZsq{}}\PY{l+s}{co}\PY{l+s}{\PYZsq{}}\PY{p}{,} \PY{n}{label}\PY{o}{=}\PY{l+s}{\PYZsq{}}\PY{l+s}{jun22}\PY{l+s}{\PYZsq{}}\PY{p}{)}
        \PY{n}{ax}\PY{o}{.}\PY{n}{plot}\PY{p}{(}\PY{n}{lats}\PY{p}{,} \PY{n}{jul15}\PY{p}{,} \PY{l+s}{\PYZsq{}}\PY{l+s}{bo}\PY{l+s}{\PYZsq{}}\PY{p}{,} \PY{n}{label}\PY{o}{=}\PY{l+s}{\PYZsq{}}\PY{l+s}{jul15}\PY{l+s}{\PYZsq{}}\PY{p}{)}
        \PY{n}{ax}\PY{o}{.}\PY{n}{plot}\PY{p}{(}\PY{l+m+mi}{56}\PY{p}{,} \PY{n}{KA\PYZus{}56N\PYZus{}jun22}\PY{p}{,} \PY{l+s}{\PYZsq{}}\PY{l+s}{cx}\PY{l+s}{\PYZsq{}}\PY{p}{,} \PY{n}{label}\PY{o}{=}\PY{l+s}{\PYZsq{}}\PY{l+s}{jun22 @ 56N}\PY{l+s}{\PYZsq{}}\PY{p}{)}
        \PY{n}{ax}\PY{o}{.}\PY{n}{plot}\PY{p}{(}\PY{l+m+mi}{56}\PY{p}{,} \PY{n}{KA\PYZus{}56N\PYZus{}jul15}\PY{p}{,} \PY{l+s}{\PYZsq{}}\PY{l+s}{bx}\PY{l+s}{\PYZsq{}}\PY{p}{,} \PY{n}{label}\PY{o}{=}\PY{l+s}{\PYZsq{}}\PY{l+s}{jul15 @ 56N}\PY{l+s}{\PYZsq{}}\PY{p}{)}
        \PY{n}{ax}\PY{o}{.}\PY{n}{plot}\PY{p}{(}\PY{l+m+mi}{56}\PY{p}{,} \PY{n}{KA\PYZus{}56N\PYZus{}jul2}\PY{p}{,} \PY{l+s}{\PYZsq{}}\PY{l+s}{kx}\PY{l+s}{\PYZsq{}}\PY{p}{,} \PY{n}{label}\PY{o}{=}\PY{l+s}{\PYZsq{}}\PY{l+s}{jul2 @ 56N}\PY{l+s}{\PYZsq{}}\PY{p}{)}
        \PY{n}{ax}\PY{o}{.}\PY{n}{annotate}\PY{p}{(}\PY{l+s}{\PYZsq{}}\PY{l+s}{\PYZdl{}K\PYZus{}A\PYZdl{} at 56 N on Jul 2 = }\PY{l+s+si}{\PYZpc{}.2f}\PY{l+s}{ \PYZdl{}W m\PYZca{}\PYZob{}\PYZhy{}2\PYZcb{}\PYZdl{}}\PY{l+s}{\PYZsq{}} \PY{o}{\PYZpc{}} \PY{n}{KA\PYZus{}56N\PYZus{}jul2}\PY{p}{,} 
                    \PY{n}{xy}\PY{o}{=}\PY{p}{(}\PY{l+m+mi}{56}\PY{p}{,} \PY{n}{KA\PYZus{}56N\PYZus{}jul2}\PY{p}{)}\PY{p}{,} 
                    \PY{n}{xytext}\PY{o}{=}\PY{p}{(}\PY{l+m+mi}{30}\PY{p}{,} \PY{l+m+mi}{430}\PY{p}{)}\PY{p}{,}
                    \PY{n}{arrowprops}\PY{o}{=}\PY{n+nb}{dict}\PY{p}{(}\PY{n}{facecolor}\PY{o}{=}\PY{l+s}{\PYZsq{}}\PY{l+s}{k}\PY{l+s}{\PYZsq{}}\PY{p}{,} \PY{n}{shrink}\PY{o}{=}\PY{l+m+mf}{0.05}\PY{p}{)}\PY{p}{)}
        
        \PY{n}{ax}\PY{o}{.}\PY{n}{set\PYZus{}title}\PY{p}{(}\PY{l+s}{\PYZdq{}}\PY{l+s}{Total daily solar radiation to horizontal surface at TOA}\PY{l+s}{\PYZdq{}}\PY{p}{)}
        \PY{n}{ax}\PY{o}{.}\PY{n}{set\PYZus{}xlabel}\PY{p}{(}\PY{l+s}{\PYZdq{}}\PY{l+s}{Latitude}\PY{l+s}{\PYZdq{}}\PY{p}{)}
        \PY{n}{ax}\PY{o}{.}\PY{n}{set\PYZus{}ylabel}\PY{p}{(}\PY{l+s}{r\PYZsq{}}\PY{l+s}{\PYZdl{}K\PYZus{}A (W m\PYZca{}\PYZob{}\PYZhy{}2\PYZcb{})\PYZdl{}}\PY{l+s}{\PYZsq{}}\PY{p}{)}
        \PY{n}{plt}\PY{o}{.}\PY{n}{legend}\PY{p}{(}\PY{n}{bbox\PYZus{}to\PYZus{}anchor}\PY{o}{=}\PY{p}{(}\PY{l+m+mf}{1.05}\PY{p}{,} \PY{l+m+mi}{1}\PY{p}{)}\PY{p}{,} \PY{n}{loc}\PY{o}{=}\PY{l+m+mi}{2}\PY{p}{,} \PY{n}{borderaxespad}\PY{o}{=}\PY{l+m+mf}{0.}\PY{p}{,} \PY{n}{numpoints}\PY{o}{=}\PY{l+m+mi}{1}\PY{p}{)}
        \PY{n}{plt}\PY{o}{.}\PY{n}{show}\PY{p}{(}\PY{p}{)}
        \PY{n}{fig}\PY{o}{.}\PY{n}{savefig}\PY{p}{(}\PY{l+s}{\PYZsq{}}\PY{l+s}{HW1.1i\PYZus{}fig2.png}\PY{l+s}{\PYZsq{}}\PY{p}{)}
\end{Verbatim}

    \begin{center}
    \adjustimage{max size={0.9\linewidth}{0.9\paperheight}}{Brodzik_assignment1_files/Brodzik_assignment1_8_0.png}
    \end{center}
    { \hspace*{\fill} \\}
    
    So the estimated daily average solar radiation to the top of the Earth's
atmosphere at 56\(^{\circ}\) N on July 2nd is 466.90 \(W m^{-2}\).

    \subsection{Q1. Part ii}\label{q1.-part-ii}

ii) if there are 7 hours of bright sunshine, what is the average
incoming solar radiation to the surface of a flat unvegetated field at
this location on this day?

    Eq 7. from ``Calculating Evaporation Notes'' gives the equation for mean
incoming shortwave radiation \(K\downarrow [W m^{-2}]\) as:

\(K\downarrow = K_A[a + b(\frac{n}{N})]\)

Looking at latitudes of locations in Table B, I think the closest
location would be that of Central SK, Canada from Mudiare(1985). I will
use the rainfree values for a and b.

Let:

Estimate N from Table C, values given for latitude=56N, interpolate
values from Jul1 - Jul5 to a value for Jul2, and remember to convert
(hours,minutes) to decimal hours)

    \begin{Verbatim}[commandchars=\\\{\}]
{\color{incolor}In [{\color{incolor}6}]:} \PY{n}{N} \PY{o}{=} \PY{n}{linear\PYZus{}model\PYZus{}value\PYZus{}at}\PY{p}{(}\PY{l+m+mf}{2.}\PY{p}{,} \PY{l+m+mi}{1}\PY{p}{,} \PY{l+m+mf}{17.} \PY{o}{+} \PY{p}{(}\PY{l+m+mf}{31.}\PY{o}{/}\PY{l+m+mf}{60.}\PY{p}{)}\PY{p}{,} \PY{l+m+mi}{5}\PY{p}{,} \PY{l+m+mf}{17.} \PY{o}{+} \PY{p}{(}\PY{l+m+mf}{25.}\PY{o}{/}\PY{l+m+mf}{60.}\PY{p}{)}\PY{p}{)}
        \PY{k}{print}\PY{p}{(}\PY{l+s}{\PYZdq{}}\PY{l+s}{Estimate of N at 56N on Jul 2 is: }\PY{l+s+si}{\PYZpc{}.2f}\PY{l+s}{\PYZdq{}} \PY{o}{\PYZpc{}} \PY{n}{N}\PY{p}{)}
\end{Verbatim}

    \begin{Verbatim}[commandchars=\\\{\}]
Estimate of N at 56N on Jul 2 is: 17.49
    \end{Verbatim}

    \begin{Verbatim}[commandchars=\\\{\}]
{\color{incolor}In [{\color{incolor}7}]:} \PY{n}{a} \PY{o}{=} \PY{l+m+mf}{0.27}
        \PY{n}{b} \PY{o}{=} \PY{l+m+mf}{0.47}
        \PY{n}{n} \PY{o}{=} \PY{l+m+mf}{7.}
        \PY{n}{K\PYZus{}down} \PY{o}{=} \PY{n}{KA\PYZus{}56N\PYZus{}jul2} \PY{o}{*} \PY{p}{(}\PY{n}{a} \PY{o}{+} \PY{n}{b} \PY{o}{*} \PY{p}{(}\PY{n}{n}\PY{o}{/}\PY{n}{N}\PY{p}{)}\PY{p}{)}
        \PY{n}{K\PYZus{}down}
\end{Verbatim}

            \begin{Verbatim}[commandchars=\\\{\}]
{\color{outcolor}Out[{\color{outcolor}7}]:} 213.881978746401
\end{Verbatim}
        
    So with 7 hours of direct sunshine at this location, the mean incoming
solar radiation to the surface of a flat unvegetated field at about
56\(^{\circ}\) N on July 2nd is 213.88 \(W m^{-2}\). So atmospheric
transmittance reduces the shortwave by more than half of that reaching
TOA.

    \subsection{Q1. Part iii}\label{q1.-part-iii}

iii) if the mean daily air temperature is 15°C and the mean daily
relative humidity is 70\%, what is the daily average clear sky incoming
longwave radiation and what is the actual daily average incoming
longwave radiation to this field?

From our class notes, incoming longwave clear sky radiation from the
atmosphere, \(L_{0,clear} [W m^{-2}]\), is expressed by the
Stephan-Boltzmann Law:

\(L_{0,clear} = \varepsilon_{clear}(T,e) \sigma T^4\)

Where apparent clear-sky emissivity, \(\varepsilon_{clear}\), is a
function of air temperature near the ground, \(T [K]\), and water vapour
pressure, \(e [mb]\), near the ground. Brutsaert(1975) derived an
empirical relationship for \(\varepsilon_{clear}\) as a function of
\(T\) and \(e\):

\[\varepsilon_{clear} = C(\frac{e}{T})^{1/m}\]

where:

\(T\) = air temperature, \([K]\)\\
\(e\) = vapour pressure near the ground, \([mb]\)\\
\(C\) = 1.24\\
\(m\) = 7

From lectures, I can use the Magnus Formula to approximate saturated
water vapour pressure \(e_s [hPa=mb]\), at 15-deg C:

\(e_s [mb] = 6.1094 \exp{\frac{17.625 T}{T + 243.04}}\)

where: \(T\) = air temperature \([°C] = 15.\)

    \begin{Verbatim}[commandchars=\\\{\}]
{\color{incolor}In [{\color{incolor}8}]:} \PY{n}{T\PYZus{}C} \PY{o}{=} \PY{l+m+mf}{15.} \PY{c}{\PYZsh{} air temperature, degrees C}
        \PY{n}{es\PYZus{}15C} \PY{o}{=} \PY{l+m+mf}{6.1094} \PY{o}{*} \PY{n}{np}\PY{o}{.}\PY{n}{exp}\PY{p}{(}\PY{p}{(}\PY{l+m+mf}{17.625} \PY{o}{*} \PY{n}{T\PYZus{}C}\PY{p}{)} \PY{o}{/} \PY{p}{(}\PY{n}{T\PYZus{}C} \PY{o}{+} \PY{l+m+mf}{243.04}\PY{p}{)}\PY{p}{)}
        \PY{k}{print}\PY{p}{(}\PY{l+s}{\PYZdq{}}\PY{l+s}{Saturated water vapour pressure at 15°C is }\PY{l+s+si}{\PYZpc{}.2f}\PY{l+s}{ mb}\PY{l+s}{\PYZdq{}} \PY{o}{\PYZpc{}} \PY{n}{es\PYZus{}15C}\PY{p}{)}
\end{Verbatim}

    \begin{Verbatim}[commandchars=\\\{\}]
Saturated water vapour pressure at 15°C is 17.02 mb
    \end{Verbatim}

    So water vapour pressure near the ground, \(e [mb]\), is 70\% of
\(e_s(T=15°C)\):

    \begin{Verbatim}[commandchars=\\\{\}]
{\color{incolor}In [{\color{incolor}9}]:} \PY{n}{RH} \PY{o}{=} \PY{l+m+mi}{70}
        \PY{n}{e} \PY{o}{=} \PY{n}{RH} \PY{o}{/} \PY{l+m+mf}{100.} \PY{o}{*} \PY{n}{es\PYZus{}15C}
        \PY{k}{print}\PY{p}{(}\PY{l+s}{\PYZdq{}}\PY{l+s}{Water vapour pressure near the ground is }\PY{l+s+si}{\PYZpc{}.2f}\PY{l+s}{ mb}\PY{l+s}{\PYZdq{}} \PY{o}{\PYZpc{}} \PY{n}{e}\PY{p}{)}
\end{Verbatim}

    \begin{Verbatim}[commandchars=\\\{\}]
Water vapour pressure near the ground is 11.91 mb
    \end{Verbatim}

    \begin{Verbatim}[commandchars=\\\{\}]
{\color{incolor}In [{\color{incolor}10}]:} \PY{c}{\PYZsh{} For the Brutsaert equation, temperature needs to be in Kelvins, so convert 15C to Kelvins}
         \PY{n}{T\PYZus{}K} \PY{o}{=} \PY{n}{T\PYZus{}C} \PY{o}{+} \PY{l+m+mf}{273.15}
         \PY{k}{print}\PY{p}{(}\PY{l+s}{\PYZdq{}}\PY{l+s}{Air temperature = }\PY{l+s+si}{\PYZpc{}.2f}\PY{l+s}{ K}\PY{l+s}{\PYZdq{}} \PY{o}{\PYZpc{}} \PY{n}{T\PYZus{}K}\PY{p}{)}
\end{Verbatim}

    \begin{Verbatim}[commandchars=\\\{\}]
Air temperature = 288.15 K
    \end{Verbatim}

    \begin{Verbatim}[commandchars=\\\{\}]
{\color{incolor}In [{\color{incolor}11}]:} \PY{n}{C} \PY{o}{=} \PY{l+m+mf}{1.24}
         \PY{n}{m} \PY{o}{=} \PY{l+m+mf}{7.}
         \PY{n}{e\PYZus{}clear} \PY{o}{=} \PY{n}{C} \PY{o}{*} \PY{p}{(}\PY{p}{(}\PY{n}{e} \PY{o}{/} \PY{n}{T\PYZus{}K}\PY{p}{)}\PY{o}{*}\PY{o}{*}\PY{p}{(}\PY{l+m+mf}{1.}\PY{o}{/}\PY{n}{m}\PY{p}{)}\PY{p}{)}
         \PY{k}{print}\PY{p}{(}\PY{l+s}{\PYZdq{}}\PY{l+s}{Clear sky emissivity (using Brutsaert}\PY{l+s}{\PYZsq{}}\PY{l+s}{s formula) is }\PY{l+s+si}{\PYZpc{}.2f}\PY{l+s}{\PYZdq{}} \PY{o}{\PYZpc{}} \PY{n}{e\PYZus{}clear}\PY{p}{)}
\end{Verbatim}

    \begin{Verbatim}[commandchars=\\\{\}]
Clear sky emissivity (using Brutsaert's formula) is 0.79
    \end{Verbatim}

    So using the Stephan-Boltzmann Law, let:

\(\varepsilon_{clear}(T=15°C,e=11.91mb)\) = 0.79\\
\(\sigma\) = 5.67 x \(10^{-8} W m^{-2} K^{-4}\)\\
\(T\) = 288.15 \(K\)

    \begin{Verbatim}[commandchars=\\\{\}]
{\color{incolor}In [{\color{incolor}12}]:} \PY{n}{sigma} \PY{o}{=} \PY{l+m+mf}{5.67e\PYZhy{}08}
         \PY{n}{longwave\PYZus{}clearsky} \PY{o}{=} \PY{n}{e\PYZus{}clear} \PY{o}{*} \PY{n}{sigma} \PY{o}{*} \PY{n}{T\PYZus{}K}\PY{o}{*}\PY{o}{*}\PY{l+m+mi}{4}
         \PY{k}{print}\PY{p}{(}\PY{l+s}{\PYZdq{}}\PY{l+s}{Incoming longwave clearsky }\PY{l+s+si}{\PYZpc{}.2f}\PY{l+s}{ [W/m\PYZca{}2]}\PY{l+s}{\PYZdq{}} \PY{o}{\PYZpc{}} \PY{n}{longwave\PYZus{}clearsky}\PY{p}{)}
\end{Verbatim}

    \begin{Verbatim}[commandchars=\\\{\}]
Incoming longwave clearsky 307.49 [W/m\^{}2]
    \end{Verbatim}

    So I estimate daily average clear sky incoming longwave radiation,
\(L\downarrow\), to be 307.49 \(W m^{-2}\), which is greater than the
incoming shortwave from part i).

    For the actual daily average incoming longwave radiation to this field,
since it there are only 7 hours of bright sunshine, I think the actual
atmospheric emissivity will need to be adjusted upward, to account for
increased emission from the clouds. I will use the adjustment used in
Sicart et al 2006, equation (9). The adjustment for emissivity from
clouds is:

\[\varepsilon_{cloudy} = \varepsilon_{clear}(1 + 0.44 RH - 0.18 \tau_{atm})\]

where:

\(RH\) = relative humidity (as a percent)\\
\(\tau_{atm}\) = shortwave transmissivity of the atmosphere

Shortwave atmospheric transmissivity is the ratio of my answers from
part i) and ii):

    \begin{Verbatim}[commandchars=\\\{\}]
{\color{incolor}In [{\color{incolor}13}]:} \PY{n}{tau\PYZus{}atm} \PY{o}{=} \PY{n}{K\PYZus{}down} \PY{o}{/} \PY{n}{KA\PYZus{}56N\PYZus{}jul2}
         \PY{k}{print}\PY{p}{(}\PY{l+s}{\PYZdq{}}\PY{l+s}{KA = }\PY{l+s+si}{\PYZpc{}.2f}\PY{l+s}{, K\PYZus{}down = }\PY{l+s+si}{\PYZpc{}.2f}\PY{l+s}{, tau\PYZus{}atm = }\PY{l+s+si}{\PYZpc{}.2f}\PY{l+s}{\PYZdq{}} \PY{o}{\PYZpc{}}\PY{p}{(}
                 \PY{n}{KA\PYZus{}56N\PYZus{}jul2}\PY{p}{,} \PY{n}{K\PYZus{}down}\PY{p}{,} \PY{n}{tau\PYZus{}atm}\PY{p}{)}\PY{p}{)}
\end{Verbatim}

    \begin{Verbatim}[commandchars=\\\{\}]
KA = 466.90, K\_down = 213.88, tau\_atm = 0.46
    \end{Verbatim}

    So using emissivity of the cloudy atmosphere in Stephan-Boltzmann:

    \begin{Verbatim}[commandchars=\\\{\}]
{\color{incolor}In [{\color{incolor}14}]:} \PY{n}{e\PYZus{}cloudy} \PY{o}{=} \PY{n}{e\PYZus{}clear} \PY{o}{*} \PY{p}{(}\PY{l+m+mf}{1.} \PY{o}{+} \PY{l+m+mf}{0.44} \PY{o}{*} \PY{p}{(}\PY{n}{RH}\PY{o}{/}\PY{l+m+mf}{100.}\PY{p}{)} \PY{o}{\PYZhy{}} \PY{l+m+mf}{0.18}\PY{o}{*} \PY{p}{(}\PY{n}{tau\PYZus{}atm}\PY{p}{)}\PY{p}{)}
         \PY{n}{longwave\PYZus{}cloudysky} \PY{o}{=} \PY{n}{e\PYZus{}cloudy} \PY{o}{*} \PY{n}{sigma} \PY{o}{*} \PY{n}{T\PYZus{}K}\PY{o}{*}\PY{o}{*}\PY{l+m+mi}{4}
         \PY{k}{print}\PY{p}{(}\PY{l+s}{\PYZdq{}}\PY{l+s}{Clear\PYZhy{}sky emissivity }\PY{l+s+si}{\PYZpc{}.2f}\PY{l+s}{\PYZdq{}} \PY{o}{\PYZpc{}} \PY{n}{e\PYZus{}clear}\PY{p}{)}
         \PY{k}{print}\PY{p}{(}\PY{l+s}{\PYZdq{}}\PY{l+s}{Cloudy\PYZhy{}sky emissivity }\PY{l+s+si}{\PYZpc{}.2f}\PY{l+s}{\PYZdq{}} \PY{o}{\PYZpc{}} \PY{n}{e\PYZus{}cloudy}\PY{p}{)}
         \PY{k}{print}\PY{p}{(}\PY{l+s}{\PYZdq{}}\PY{l+s}{Incoming longwave radiation }\PY{l+s+si}{\PYZpc{}.2f}\PY{l+s}{\PYZdq{}} \PY{o}{\PYZpc{}} \PY{n}{longwave\PYZus{}cloudysky}\PY{p}{)}
\end{Verbatim}

    \begin{Verbatim}[commandchars=\\\{\}]
Clear-sky emissivity 0.79
Cloudy-sky emissivity 0.96
Incoming longwave radiation 376.84
    \end{Verbatim}

    So this matches my expectation from class notes that
\(\varepsilon_{clear} < \varepsilon_{cloudy}\), and I estimate actual
daily average incoming longwave radiation to this field as 376.84
\(W m^{-2}\).

    \subsection{Q1. Part iv}\label{q1.-part-iv}

iv) if the surface temperature is 20°C and albedo is 0.15 what is the
average net radiation over the day?

The average net radiation over the day is the sum of net shortwave and
net longwave:

\[Q^\ast = K^\ast - L^\ast = K\downarrow(1 - \alpha) + L\downarrow - L\uparrow\]

where:\\
\(K^\ast\) = net shortwave\\
\(L^\ast\) = net longwave\\
\(K\downarrow\) = incoming shortwave\\
\(\alpha\) = shortwave albedo\\
\(L\downarrow\) = incoming longwave\\
\(L\uparrow\) = outgoing longwave from the terrain surface

Air temperature is not a factor for the incoming shortwave, so I can use
incoming shortwave from part ii:

\(K\downarrow = 213.88 W m^{-2}\)\\
\(\alpha = 0.15\)

    \begin{Verbatim}[commandchars=\\\{\}]
{\color{incolor}In [{\color{incolor}15}]:} \PY{n}{albedo} \PY{o}{=} \PY{l+m+mf}{0.15}
         \PY{n}{K\PYZus{}net} \PY{o}{=} \PY{n}{K\PYZus{}down} \PY{o}{*} \PY{p}{(}\PY{l+m+mf}{1.} \PY{o}{\PYZhy{}} \PY{n}{albedo}\PY{p}{)}
         \PY{n}{K\PYZus{}net}
\end{Verbatim}

            \begin{Verbatim}[commandchars=\\\{\}]
{\color{outcolor}Out[{\color{outcolor}15}]:} 181.79968193444083
\end{Verbatim}
        
    And adjust the incoming longwave from part iii for the higher
temperature:

    \begin{Verbatim}[commandchars=\\\{\}]
{\color{incolor}In [{\color{incolor}16}]:} \PY{n}{T\PYZus{}C} \PY{o}{=} \PY{l+m+mf}{20.} \PY{c}{\PYZsh{} air temperature, degrees C}
         \PY{n}{T\PYZus{}K} \PY{o}{=} \PY{n}{T\PYZus{}C} \PY{o}{+} \PY{l+m+mf}{273.15}
         \PY{n}{e\PYZus{}clear\PYZus{}20} \PY{o}{=} \PY{n}{C} \PY{o}{*} \PY{p}{(}\PY{p}{(}\PY{n}{e} \PY{o}{/} \PY{n}{T\PYZus{}K}\PY{p}{)}\PY{o}{*}\PY{o}{*}\PY{p}{(}\PY{l+m+mf}{1.}\PY{o}{/}\PY{n}{m}\PY{p}{)}\PY{p}{)} \PY{c}{\PYZsh{} clear sky emission at T=20}
         \PY{n}{e\PYZus{}cloudy\PYZus{}20} \PY{o}{=} \PY{n}{e\PYZus{}clear\PYZus{}20} \PY{o}{*} \PY{p}{(}\PY{l+m+mf}{1.} \PY{o}{+} \PY{l+m+mf}{0.44} \PY{o}{*} \PY{p}{(}\PY{n}{RH}\PY{o}{/}\PY{l+m+mf}{100.}\PY{p}{)} \PY{o}{\PYZhy{}} \PY{l+m+mf}{0.18}\PY{o}{*} \PY{p}{(}\PY{n}{tau\PYZus{}atm}\PY{p}{)}\PY{p}{)}
         \PY{n}{longwave\PYZus{}cloudysky\PYZus{}20} \PY{o}{=} \PY{n}{e\PYZus{}cloudy\PYZus{}20} \PY{o}{*} \PY{n}{sigma} \PY{o}{*} \PY{n}{T\PYZus{}K}\PY{o}{*}\PY{o}{*}\PY{l+m+mi}{4} \PY{c}{\PYZsh{} W/m\PYZca{}2}
         \PY{k}{print}\PY{p}{(}\PY{l+s}{\PYZdq{}}\PY{l+s}{Incoming longwave radiation for T=20°C is now }\PY{l+s+si}{\PYZpc{}.2f}\PY{l+s}{ W/m\PYZca{}2}\PY{l+s}{\PYZdq{}} \PY{o}{\PYZpc{}} \PY{n}{longwave\PYZus{}cloudysky\PYZus{}20}\PY{p}{)}
\end{Verbatim}

    \begin{Verbatim}[commandchars=\\\{\}]
Incoming longwave radiation for T=20°C is now 402.69 W/m\^{}2
    \end{Verbatim}

    Outgoing longwave is longwave emission from the terrain surface. I use
the Stephan-Boltzmann equation, assuming albedo of the terrain is close
to 1, say 0.98, and that the air temperature is an adequate estimate for
the surface temperature:

    \begin{Verbatim}[commandchars=\\\{\}]
{\color{incolor}In [{\color{incolor}17}]:} \PY{n}{surface\PYZus{}emissivity} \PY{o}{=} \PY{l+m+mf}{0.98}
         \PY{n}{longwave\PYZus{}up} \PY{o}{=} \PY{l+m+mf}{0.98} \PY{o}{*} \PY{n}{sigma} \PY{o}{*} \PY{n}{T\PYZus{}K}\PY{o}{*}\PY{o}{*}\PY{l+m+mi}{4} \PY{c}{\PYZsh{} W/m\PYZca{}2}
         \PY{k}{print}\PY{p}{(}\PY{l+s}{\PYZdq{}}\PY{l+s}{Outgoing longwave radiation from surface for T=20°C is }\PY{l+s+si}{\PYZpc{}.2f}\PY{l+s}{ W/m\PYZca{}2}\PY{l+s}{\PYZdq{}} \PY{o}{\PYZpc{}}
               \PY{n}{longwave\PYZus{}up}\PY{p}{)}
\end{Verbatim}

    \begin{Verbatim}[commandchars=\\\{\}]
Outgoing longwave radiation from surface for T=20°C is 410.36 W/m\^{}2
    \end{Verbatim}

    \begin{Verbatim}[commandchars=\\\{\}]
{\color{incolor}In [{\color{incolor}18}]:} \PY{n}{net} \PY{o}{=} \PY{n}{K\PYZus{}net} \PY{o}{+} \PY{n}{longwave\PYZus{}cloudysky\PYZus{}20} \PY{o}{\PYZhy{}} \PY{n}{longwave\PYZus{}up}
         \PY{k}{print}\PY{p}{(}\PY{n}{K\PYZus{}net}\PY{p}{,} \PY{n}{longwave\PYZus{}cloudysky\PYZus{}20}\PY{p}{,} \PY{n}{longwave\PYZus{}up}\PY{p}{)}
         \PY{k}{print}\PY{p}{(}\PY{l+s}{\PYZdq{}}\PY{l+s}{Net\PYZus{}radiation for T=20 is }\PY{l+s+si}{\PYZpc{}.2f}\PY{l+s}{\PYZdq{}} \PY{o}{\PYZpc{}} \PY{n}{net}\PY{p}{)}
\end{Verbatim}

    \begin{Verbatim}[commandchars=\\\{\}]
(181.79968193444083, 402.69262523294549, 410.3635032136096)
Net\_radiation for T=20 is 174.13
    \end{Verbatim}

    So I estimate average net radiation in this location over the day to be
159.67 \(W m^{-2}\).

    \subsection{Q1. Part v}\label{q1.-part-v}

v) what would the daily average incoming solar radiation and incoming
longwave radiation fluxes be to the sub-canopy floor under an adjacent
pine canopy at this location on this day, if the pine canopy had a leaf
area index of 2.1 and a sky view factor of 0.2? If the sub-canopy
surface temperature and albedo are the same as in the field, what is the
sub-canopy net radiation?

Per Pomeroy and Dion, 1996, the transmittance through a forest canopy
can be modeled with an extinction coefficient:

\[\tau = exp^{\frac{-Q_{ext}\ {LAI}'}{\sin\theta}}\]

where:

\(\tau\) = transmittance through forest canopy\\
\(Q_{ext}\) = extinction efficiency (dimensionless)\\
\(LAI'\) = effective winter leaf area index\\
\(\theta\) = solar angle above the horizon

From Pomeroy lecture on Day 1, slide 42, apparently \(Q_{ext}\) cancels
\(\sin\theta\) , so \(Q_{ext} \approx \sin\theta\) so their ratio will
\$\rightarrow\$1. and \(\tau\) is:

    \begin{Verbatim}[commandchars=\\\{\}]
{\color{incolor}In [{\color{incolor}19}]:} \PY{n}{eff\PYZus{}lai} \PY{o}{=} \PY{l+m+mf}{2.1}
         \PY{n}{forest\PYZus{}tau} \PY{o}{=} \PY{n}{exp}\PY{p}{(}\PY{o}{\PYZhy{}}\PY{l+m+mi}{1} \PY{o}{*} \PY{n}{eff\PYZus{}lai}\PY{p}{)}
         \PY{n}{forest\PYZus{}tau}
\end{Verbatim}

            \begin{Verbatim}[commandchars=\\\{\}]
{\color{outcolor}Out[{\color{outcolor}19}]:} 0.12245642825298191
\end{Verbatim}
        
    So I estimate \(\tau\) = 0.12. This is consistent with my notes from
class, which indicate that a good rule of thumb for \(\tau\) in forests
is 0.1.

For daily average incoming solar radiation and incoming longwave
radiation fluxes to the sub-canopy floor under an adjacent pine canopy
at this location on this day, incoming shortwave will be \(K^\ast\)
times the transmittance, and net shortwave will be incoming shortwave
times \((1 - \alpha)\):

    \begin{Verbatim}[commandchars=\\\{\}]
{\color{incolor}In [{\color{incolor}20}]:} \PY{n}{subcanopy\PYZus{}shortwave\PYZus{}down} \PY{o}{=} \PY{n}{K\PYZus{}net} \PY{o}{*} \PY{n}{forest\PYZus{}tau}
         \PY{n}{net\PYZus{}subcanopy\PYZus{}shortwave} \PY{o}{=} \PY{n}{subcanopy\PYZus{}shortwave\PYZus{}down} \PY{o}{*} \PY{p}{(}\PY{l+m+mf}{1.} \PY{o}{\PYZhy{}} \PY{n}{albedo}\PY{p}{)}
         \PY{k}{print}\PY{p}{(}\PY{l+s}{\PYZdq{}}\PY{l+s}{shortwave down to subcanopy }\PY{l+s+si}{\PYZpc{}.2f}\PY{l+s}{ W/m\PYZca{}2}\PY{l+s}{\PYZdq{}} \PY{o}{\PYZpc{}} \PY{n}{subcanopy\PYZus{}shortwave\PYZus{}down}\PY{p}{)}
         \PY{k}{print}\PY{p}{(}\PY{l+s}{\PYZdq{}}\PY{l+s}{net subcanopy shortwave }\PY{l+s+si}{\PYZpc{}.2f}\PY{l+s}{ W/m\PYZca{}2}\PY{l+s}{\PYZdq{}} \PY{o}{\PYZpc{}} \PY{n}{net\PYZus{}subcanopy\PYZus{}shortwave}\PY{p}{)}
\end{Verbatim}

    \begin{Verbatim}[commandchars=\\\{\}]
shortwave down to subcanopy 22.26 W/m\^{}2
net subcanopy shortwave 18.92 W/m\^{}2
    \end{Verbatim}

    The incoming longwave to the subcanopy will come from a combination of
the sky and the trees. Sicart et al. 2006 (equation 6) found that
terrain and vegetation emissions of longwave radiation could be
represented as separate components weighted by sky view factor:

\[L = V_f L_0 + (1 - V_f) \varepsilon_s \sigma T_{s}^4\]

let:\\
\(V_f\) = 0.2 (given)\\
\(L_0\) = longwave from the cloudysky at 20°C from part iv,
\(W m^{-2}\)\\
\(\varepsilon_s\) = 0.98 emissivity of the forest, assume it is close to
1, from Sicart ``terrain emissivity is close to 1 for snow and most
natural surfaces''\\
\(T_{s}\) = 20 + 273.15 \(K\) (same temperature as surrounding field)

The outgoing longwave from the subcanopy floor will be the same as that
from the surrounding field.

    \begin{Verbatim}[commandchars=\\\{\}]
{\color{incolor}In [{\color{incolor}21}]:} \PY{n}{sky\PYZus{}view\PYZus{}factor} \PY{o}{=} \PY{l+m+mf}{0.2}
         \PY{n}{subcanopy\PYZus{}longwave\PYZus{}down} \PY{o}{=} \PY{n}{sky\PYZus{}view\PYZus{}factor} \PY{o}{*} \PY{n}{longwave\PYZus{}cloudysky\PYZus{}20} \PY{o}{+} \PYZbs{}
                \PY{p}{(}\PY{l+m+mf}{1.} \PY{o}{\PYZhy{}} \PY{n}{sky\PYZus{}view\PYZus{}factor}\PY{p}{)} \PY{o}{*} \PY{n}{surface\PYZus{}emissivity} \PY{o}{*} \PY{n}{sigma} \PY{o}{*} \PY{n}{T\PYZus{}K}\PY{o}{*}\PY{o}{*}\PY{l+m+mi}{4} 
         \PY{n}{net\PYZus{}subcanopy\PYZus{}longwave} \PY{o}{=} \PY{n}{subcanopy\PYZus{}longwave\PYZus{}down} \PY{o}{\PYZhy{}} \PY{n}{longwave\PYZus{}up}
         \PY{k}{print}\PY{p}{(}\PY{l+s}{\PYZdq{}}\PY{l+s}{longwave down to subcanopy }\PY{l+s+si}{\PYZpc{}.2f}\PY{l+s}{ W/m\PYZca{}2}\PY{l+s}{\PYZdq{}} \PY{o}{\PYZpc{}} \PY{n}{subcanopy\PYZus{}longwave\PYZus{}down}\PY{p}{)}
         \PY{k}{print}\PY{p}{(}\PY{l+s}{\PYZdq{}}\PY{l+s}{longwave up from subcanopy }\PY{l+s+si}{\PYZpc{}.2f}\PY{l+s}{ W/m\PYZca{}2}\PY{l+s}{\PYZdq{}} \PY{o}{\PYZpc{}} \PY{n}{longwave\PYZus{}up}\PY{p}{)}
         \PY{k}{print}\PY{p}{(}\PY{l+s}{\PYZdq{}}\PY{l+s}{net subcanopy longwave }\PY{l+s+si}{\PYZpc{}.2f}\PY{l+s}{ W/m\PYZca{}2}\PY{l+s}{\PYZdq{}} \PY{o}{\PYZpc{}} \PY{n}{net\PYZus{}subcanopy\PYZus{}longwave}\PY{p}{)}
\end{Verbatim}

    \begin{Verbatim}[commandchars=\\\{\}]
longwave down to subcanopy 408.83 W/m\^{}2
longwave up from subcanopy 410.36 W/m\^{}2
net subcanopy longwave -1.53 W/m\^{}2
    \end{Verbatim}

    So the subcanopy is losing longwave radiation.

    \begin{Verbatim}[commandchars=\\\{\}]
{\color{incolor}In [{\color{incolor}22}]:} \PY{n}{net\PYZus{}subcanopy} \PY{o}{=} \PY{n}{net\PYZus{}subcanopy\PYZus{}shortwave} \PY{o}{+} \PY{n}{net\PYZus{}subcanopy\PYZus{}longwave}
\end{Verbatim}

    Finally, the subcanopy net radiation \(Q^\ast = K^\ast + L^\ast\) =
14.50 \(W m^{-1}\).

    \subsection{Question 2}\label{question-2}

Use equations presented in the paper by Harder and Pomeroy (2013) to
estimate precipitation phase:

During a precipitation measurement of 5 mm over an hour into a remote
unattended Alter-shielded weighing precipitation gauge in a forested
clearing you measure an air temperature of +1.0 C, RH of 60\%, with very
low wind speed. What is the water equivalent depth of snowfall? What is
the depth of rainfall?

Harder and Pomeroy (2013) derive a psychrometric equation to derive
phase change, expressed as rainfall fraction, \(f_r\), as a sigmoidal
function:

\[f_r(T_i) = \frac{1}{1 + b * c^{T_{i}}}\]

where:\\
\(b\), \(c\) = best fit coefficients, both are functions of time scale
of measurement (15-min, hourly, daily)\\
\(T_i\) = hydrometeor temperature

An iterative solution for \(T_i\), Appendix eq. A.5, is:

\[T_i = T_a + \frac{D}{\lambda_t}  L (\rho_{T_a} - \rho_{sat(T_i)})\]

where:\\
\(T_a\) = air temperature \([K]\)\\
\(D\) = diffusivity of water vapour in air \([m^2 s^{-1}]\)\\
\(\lambda_t\) = thermal conductivity of air
\([J\ m^{-1} s^{-1} K^{-1}]\)\\
\(L\) = latent heat of vaporisation or sublimation \([J\ kg^{-1}]\)\\
\(\rho_{T_a}\) = water vapour density in the free atmosphere
\([kg\ m^{-3}]\)\\
\(\rho_{sat(T_i)}\) = saturated water vapour density at the hydrometeor
surface \([kg\ m^{-3}]\)

Since the wind speed is low, I will assume thermodynamic equilibrium.

I am given \(T_a\) = 1.0 °\(C\) = 274.15 \(K\). For each of the next
terms:

\(D\): Following the appendix equation A.6 (Thorpe and Mason, 1966), I
can estimate \(D [m^2 s^{-1}]\) as function of \(T_a [K]\):

\[D = 2.06 * 10^{-5} * \left(\frac{T_a}{273.15}\right)^{1.75}\]

    \begin{Verbatim}[commandchars=\\\{\}]
{\color{incolor}In [{\color{incolor}23}]:} \PY{n}{C\PYZus{}to\PYZus{}K\PYZus{}offset} \PY{o}{=} \PY{l+m+mf}{273.15}
         \PY{n}{Tair\PYZus{}C} \PY{o}{=} \PY{l+m+mf}{1.0}  
         \PY{n}{Tair\PYZus{}K} \PY{o}{=} \PY{n}{Tair\PYZus{}C} \PY{o}{+} \PY{n}{C\PYZus{}to\PYZus{}K\PYZus{}offset}
         \PY{n}{D\PYZus{}m2ps} \PY{o}{=} \PY{l+m+mf}{2.06e\PYZhy{}5} \PY{o}{*} \PY{p}{(}\PY{n}{Tair\PYZus{}K}\PY{o}{/}\PY{l+m+mf}{273.15}\PY{p}{)}\PY{o}{*}\PY{o}{*}\PY{l+m+mf}{1.75}
         \PY{n}{C\PYZus{}to\PYZus{}K\PYZus{}offset}\PY{p}{,} \PY{n}{Tair\PYZus{}K}\PY{p}{,} \PY{n}{D\PYZus{}m2ps}
\end{Verbatim}

            \begin{Verbatim}[commandchars=\\\{\}]
{\color{outcolor}Out[{\color{outcolor}23}]:} (273.15, 274.15, 2.0732159900990022e-05)
\end{Verbatim}
        
    \(\lambda_t\): Following appendix equation A.9 (List, 1949), I can
estimate \(\lambda_t\ [J m^{-1} s^{-1} K^{-1}]\) as function of
\(T_a [K]\):

\[\lambda_t = 0.000063 * T_a + 0.00673\]

    \begin{Verbatim}[commandchars=\\\{\}]
{\color{incolor}In [{\color{incolor}24}]:} \PY{n}{lambda\PYZus{}t} \PY{o}{=} \PY{l+m+mf}{0.000063} \PY{o}{*} \PY{n}{Tair\PYZus{}K} \PY{o}{+} \PY{l+m+mf}{0.00673}
         \PY{k}{print}\PY{p}{(}\PY{n}{D\PYZus{}m2ps}\PY{p}{,} \PY{n}{lambda\PYZus{}t}\PY{p}{,} \PY{n}{D\PYZus{}m2ps}\PY{o}{/}\PY{n}{lambda\PYZus{}t}\PY{p}{)}
\end{Verbatim}

    \begin{Verbatim}[commandchars=\\\{\}]
(2.0732159900990022e-05, 0.024001449999999997, 0.0008637878086944757)
    \end{Verbatim}

    (I wasn't sure about temperature units in \(C\) or \(K\), but my value
for the psychrometric exchange ratio, \(D/\lambda_t\), at 0 °C is
consistent with the plot in Figure A1(b)).

\(L\): Since the temperature is \textgreater{} 0 °C, I will use heat of
vaporization, \(L_v [J\ kg^{-1}]\), equation A.11, as a function of
\(T_a [C]\):

\[L_v = 1000(2501 - (2.361 T))\]

    \begin{Verbatim}[commandchars=\\\{\}]
{\color{incolor}In [{\color{incolor}25}]:} \PY{n}{Lv\PYZus{}Jpkg} \PY{o}{=} \PY{l+m+mf}{1000.} \PY{o}{*} \PY{p}{(}\PY{l+m+mi}{2501} \PY{o}{\PYZhy{}} \PY{p}{(}\PY{l+m+mf}{2.361} \PY{o}{*} \PY{n}{Tair\PYZus{}C}\PY{p}{)}\PY{p}{)}
         \PY{k}{print}\PY{p}{(}\PY{l+s}{\PYZdq{}}\PY{l+s}{Latent heat of vaporization = }\PY{l+s+si}{\PYZpc{}.2E}\PY{l+s}{ J/kg}\PY{l+s}{\PYZdq{}} \PY{o}{\PYZpc{}} \PY{n}{Lv\PYZus{}Jpkg}\PY{p}{)}
\end{Verbatim}

    \begin{Verbatim}[commandchars=\\\{\}]
Latent heat of vaporization = 2.50E+06 J/kg
    \end{Verbatim}

    \(\rho_{T_a}\), \(\rho_{sat(T_s)}\) : To estimate water vapour density
\([kg\ m^{-3}]\) in the free atmosphere, I calculate the water vapour
pressure, e \([Pa]\), using Dingman (2015), eq 3.9a, p.113:

\[e = \frac{RH}{100} * 611\ exp\left(\frac{17.27\ T}{T + 237.3}\right)\]

where:\\
RH = relative humidity = 70\%\\
T = air temperature, \(°C\)

    \begin{Verbatim}[commandchars=\\\{\}]
{\color{incolor}In [{\color{incolor}26}]:} \PY{n}{RH} \PY{o}{=} \PY{l+m+mf}{60.}
         \PY{n}{saturated\PYZus{}vapour\PYZus{}pressure\PYZus{}Pa} \PY{o}{=} \PY{l+m+mi}{611} \PY{o}{*} \PY{n}{np}\PY{o}{.}\PY{n}{exp}\PY{p}{(}\PY{p}{(}\PY{l+m+mf}{17.27} \PY{o}{*} \PY{n}{Tair\PYZus{}C}\PY{p}{)}\PY{o}{/}\PY{p}{(}\PY{n}{Tair\PYZus{}C} \PY{o}{+} \PY{l+m+mf}{237.3}\PY{p}{)}\PY{p}{)}
         \PY{n}{free\PYZus{}air\PYZus{}vapour\PYZus{}pressure\PYZus{}Pa} \PY{o}{=} \PY{p}{(}\PY{n}{RH} \PY{o}{/} \PY{l+m+mf}{100.}\PY{p}{)} \PY{o}{*} \PY{n}{saturated\PYZus{}vapour\PYZus{}pressure\PYZus{}Pa}
         \PY{k}{print}\PY{p}{(}\PY{l+s}{\PYZdq{}}\PY{l+s}{saturated vapour pressure [Pa] = }\PY{l+s+si}{\PYZpc{}.2f}\PY{l+s}{ Pa}\PY{l+s}{\PYZdq{}} \PY{o}{\PYZpc{}} \PY{p}{(}\PY{n}{saturated\PYZus{}vapour\PYZus{}pressure\PYZus{}Pa}\PY{p}{)}\PY{p}{)}
         \PY{k}{print}\PY{p}{(}\PY{l+s}{\PYZdq{}}\PY{l+s}{free air vapour pressure [Pa] for }\PY{l+s+si}{\PYZpc{}d}\PY{l+s+si}{\PYZpc{}\PYZpc{}}\PY{l+s}{ RH = }\PY{l+s+si}{\PYZpc{}.2f}\PY{l+s}{ Pa}\PY{l+s}{\PYZdq{}} \PY{o}{\PYZpc{}} \PY{p}{(}
                 \PY{n}{RH}\PY{p}{,} \PY{n}{free\PYZus{}air\PYZus{}vapour\PYZus{}pressure\PYZus{}Pa}\PY{p}{)}\PY{p}{)}
\end{Verbatim}

    \begin{Verbatim}[commandchars=\\\{\}]
saturated vapour pressure [Pa] = 656.92 Pa
free air vapour pressure [Pa] for 60\% RH = 394.15 Pa
    \end{Verbatim}

    And then convert pressure to density using the ideal gas law (also from
Appendix A.8):

\[\rho = \frac{m_we}{RT}\]

where:\\
\(m_w\) = molecular weight of water = 0.01801528 \(kg\ mol^{-1}\)\\
\(e\) = water vapour pressure, \(kPa\)\\
\(R\) = Universal Gas Constant 8.31441 \(J\ mol^{-1}\ K^{-1}\)\\
\(T\) = temperature \(K\)

    \begin{Verbatim}[commandchars=\\\{\}]
{\color{incolor}In [{\color{incolor}27}]:} \PY{n}{mw} \PY{o}{=} \PY{l+m+mf}{0.01801528}  \PY{c}{\PYZsh{} kg mol\PYZca{}\PYZhy{}1}
         \PY{n}{R} \PY{o}{=} \PY{l+m+mf}{8.31441}      \PY{c}{\PYZsh{} J mol\PYZca{}\PYZhy{}1 K\PYZca{}\PYZhy{}1}
         \PY{n}{Pa\PYZus{}per\PYZus{}kPa} \PY{o}{=} \PY{l+m+mf}{1000.}  \PY{c}{\PYZsh{} conversion}
         \PY{n}{saturated\PYZus{}vapour\PYZus{}density\PYZus{}kgpm3} \PY{o}{=} \PY{p}{(}
             \PY{n}{mw} \PY{o}{*} \PY{n}{saturated\PYZus{}vapour\PYZus{}pressure\PYZus{}Pa} \PY{o}{/} \PY{n}{Pa\PYZus{}per\PYZus{}kPa}\PY{p}{)} \PY{o}{/} \PY{p}{(}\PY{n}{R} \PY{o}{*} \PY{n}{Tair\PYZus{}K}\PY{p}{)}
         \PY{n}{free\PYZus{}air\PYZus{}vapour\PYZus{}density\PYZus{}kgpm3} \PY{o}{=} \PY{p}{(}
             \PY{n}{mw} \PY{o}{*} \PY{n}{free\PYZus{}air\PYZus{}vapour\PYZus{}pressure\PYZus{}Pa} \PY{o}{/} \PY{n}{Pa\PYZus{}per\PYZus{}kPa}\PY{p}{)} \PY{o}{/} \PY{p}{(}\PY{n}{R} \PY{o}{*} \PY{n}{Tair\PYZus{}K}\PY{p}{)}
         \PY{k}{print}\PY{p}{(}\PY{l+s}{\PYZdq{}}\PY{l+s}{saturated water vapour density = }\PY{l+s+si}{\PYZpc{}.2E}\PY{l+s}{ kg m\PYZca{}\PYZhy{}3}\PY{l+s}{\PYZdq{}} \PY{o}{\PYZpc{}} \PY{p}{(}\PY{n}{saturated\PYZus{}vapour\PYZus{}density\PYZus{}kgpm3}\PY{p}{)}\PY{p}{)}
         \PY{k}{print}\PY{p}{(}\PY{l+s}{\PYZdq{}}\PY{l+s}{free air water vapour density for }\PY{l+s+si}{\PYZpc{}d}\PY{l+s}{ }\PY{l+s+si}{\PYZpc{}\PYZpc{}}\PY{l+s}{ RH = }\PY{l+s+si}{\PYZpc{}.2E}\PY{l+s}{ kg m\PYZca{}\PYZhy{}3}\PY{l+s}{\PYZdq{}} \PY{o}{\PYZpc{}} \PY{p}{(}
                 \PY{n}{RH}\PY{p}{,} \PY{n}{free\PYZus{}air\PYZus{}vapour\PYZus{}density\PYZus{}kgpm3}\PY{p}{)}\PY{p}{)}
\end{Verbatim}

    \begin{Verbatim}[commandchars=\\\{\}]
saturated water vapour density = 5.19E-06 kg m\^{}-3
free air water vapour density for 60 \% RH = 3.12E-06 kg m\^{}-3
    \end{Verbatim}

    Solving for hydrometeor temperature:

    \begin{Verbatim}[commandchars=\\\{\}]
{\color{incolor}In [{\color{incolor}28}]:} \PY{n}{T\PYZus{}i} \PY{o}{=} \PY{n}{Tair\PYZus{}C} \PY{o}{+} \PY{p}{(}\PY{n}{D\PYZus{}m2ps}\PY{o}{/}\PY{n}{lambda\PYZus{}t}\PY{p}{)} \PY{o}{*} \PY{n}{Lv\PYZus{}Jpkg} \PY{o}{*} \PY{p}{(}
             \PY{n}{free\PYZus{}air\PYZus{}vapour\PYZus{}density\PYZus{}kgpm3} \PY{o}{\PYZhy{}} \PY{n}{saturated\PYZus{}vapour\PYZus{}density\PYZus{}kgpm3}\PY{p}{)}
         \PY{k}{print}\PY{p}{(}\PY{l+s}{\PYZdq{}}\PY{l+s}{hydrometeor temperature = }\PY{l+s+si}{\PYZpc{}.2E}\PY{l+s}{ deg\PYZhy{}C}\PY{l+s}{\PYZdq{}} \PY{o}{\PYZpc{}} \PY{n}{T\PYZus{}i}\PY{p}{)}
\end{Verbatim}

    \begin{Verbatim}[commandchars=\\\{\}]
hydrometeor temperature = 9.96E-01 deg-C
    \end{Verbatim}

    And finally, using b and c from middle plot (hourly) of Fig. 6:

    \begin{Verbatim}[commandchars=\\\{\}]
{\color{incolor}In [{\color{incolor}29}]:} \PY{n}{b} \PY{o}{=} \PY{l+m+mf}{2.50286}
         \PY{n}{c} \PY{o}{=} \PY{l+m+mf}{0.125006}
         \PY{n}{rainfall\PYZus{}fraction} \PY{o}{=} \PY{l+m+mf}{1.} \PY{o}{/} \PY{p}{(}\PY{l+m+mf}{1.} \PY{o}{+} \PY{p}{(}\PY{n}{b} \PY{o}{*} \PY{n}{c}\PY{o}{*}\PY{o}{*}\PY{n}{T\PYZus{}i}\PY{p}{)}\PY{p}{)}
         \PY{n}{precip\PYZus{}mm} \PY{o}{=} \PY{l+m+mi}{5}
         \PY{k}{print}\PY{p}{(}\PY{n}{rainfall\PYZus{}fraction}\PY{p}{,} \PY{n}{rainfall\PYZus{}fraction} \PY{o}{*} \PY{n}{precip\PYZus{}mm}\PY{p}{,} \PYZbs{}
               \PY{p}{(}\PY{p}{(}\PY{l+m+mi}{1} \PY{o}{\PYZhy{}} \PY{n}{rainfall\PYZus{}fraction}\PY{p}{)} \PY{o}{*} \PY{n}{precip\PYZus{}mm}\PY{p}{)}\PY{p}{)}
                                   
\end{Verbatim}

    \begin{Verbatim}[commandchars=\\\{\}]
(0.75999258631298428, 3.7999629315649215, 1.2000370684350785)
    \end{Verbatim}

    So given a gauge precip measurement of 5 \(mm\), I estimate it fell as
3.8 \(mm\) of rainfall, and 1.2 \(mm\) water equivalent from snowfall.

    \subsection{Question 3}\label{question-3}

You are measuring air temperature and water vapour content over a
natural grassland (roughness length = 0.03 m) to get an idea of how
available energy is being used at this site. You have installed the
minimum of two levels of measurements of wind speed, temperature and
humidity. The daily average measurements are (heights above the ground):
- 1-m height, wind speed is 0.6 \(m\ s^{-1}\), air temperature 20
\(^oC\), and vapour pressure 2.0 \(kPa\);\\
- 2-m height, wind speed is 0.62 \(m\ s^{-1}\), air temperature 19
\(^oC\), and vapour pressure 1.5 \(kPa\).

Assume Pa = 101.3 kPa, \(\rho_a\) = 1.2 \(kg\ m^{-3}\), and \(c_p\) =
1005 \(J\ kg^{-1} K^{-1}\) and \(L_v\) = 2.454 \(MJ\ kg^{-1}\).\\
(a) Estimate a \(z_0\) for momentum, heat and water vapour from the
vegetation height. Then using bulk transfer flux-gradient calculations
(example in Helgason and Pomeroy, 1995) and ignoring stability
corrections find \(Q_E\) and \(Q_H\).\\
(b) Calculate the footprint representative of these measurements which
accounts for 80\% of the flux. Recall that you will first need to
calculate u*.

Per Helgason and Pomeroy, 2005, a typical method to estimate momentum
roughness length \(z_{0m}\) is to plot \(ln(z)\) vs. \(\bar{u}\) for
neutral conditions and then determine the value of \(z_{0m}\) as the
y-intercept where \(\bar{u}\) =0.

    \begin{Verbatim}[commandchars=\\\{\}]
{\color{incolor}In [{\color{incolor}62}]:} \PY{k+kn}{import} \PY{n+nn}{numpy} \PY{k+kn}{as} \PY{n+nn}{np}
         
         \PY{n}{u1} \PY{o}{=} \PY{l+m+mf}{0.6}  \PY{c}{\PYZsh{} m/s}
         \PY{n}{u2} \PY{o}{=} \PY{l+m+mf}{0.62} \PY{c}{\PYZsh{} m/s}
         \PY{n}{z1} \PY{o}{=} \PY{l+m+mf}{1.0}  \PY{c}{\PYZsh{} m}
         \PY{n}{z2} \PY{o}{=} \PY{l+m+mf}{2.0}  \PY{c}{\PYZsh{} m}
         
         \PY{c}{\PYZsh{} quick linear interpolation function}
         \PY{c}{\PYZsh{} returns slope and intercept}
         \PY{k}{def} \PY{n+nf}{linear\PYZus{}model\PYZus{}parms}\PY{p}{(}\PY{n}{x1}\PY{p}{,} \PY{n}{y1}\PY{p}{,} \PY{n}{x2}\PY{p}{,} \PY{n}{y2}\PY{p}{)}\PY{p}{:}
             \PY{n}{slope} \PY{o}{=} \PY{p}{(}\PY{n}{y2} \PY{o}{\PYZhy{}} \PY{n}{y1}\PY{p}{)} \PY{o}{/} \PY{p}{(}\PY{n}{x2} \PY{o}{\PYZhy{}} \PY{n}{x1}\PY{p}{)}
             \PY{c}{\PYZsh{} y = mx + b ==\PYZgt{} b = y \PYZhy{} mx}
             \PY{n}{intercept} \PY{o}{=}  \PY{n}{y1} \PY{o}{\PYZhy{}} \PY{p}{(}\PY{n}{slope} \PY{o}{*} \PY{n}{x1}\PY{p}{)}
             \PY{k}{return} \PY{p}{\PYZob{}}\PY{l+s}{\PYZsq{}}\PY{l+s}{m}\PY{l+s}{\PYZsq{}}\PY{p}{:}\PY{n}{slope}\PY{p}{,} \PY{l+s}{\PYZsq{}}\PY{l+s}{b}\PY{l+s}{\PYZsq{}}\PY{p}{:}\PY{n}{intercept}\PY{p}{\PYZcb{}}
         
         \PY{n}{parms} \PY{o}{=} \PY{n}{linear\PYZus{}model\PYZus{}parms}\PY{p}{(}\PY{n}{u1}\PY{p}{,} \PY{n}{np}\PY{o}{.}\PY{n}{log}\PY{p}{(}\PY{n}{z1}\PY{p}{)}\PY{p}{,} \PY{n}{u2}\PY{p}{,} \PY{n}{np}\PY{o}{.}\PY{n}{log}\PY{p}{(}\PY{n}{z2}\PY{p}{)}\PY{p}{)}
         \PY{k}{print}\PY{p}{(}\PY{l+s}{\PYZdq{}}\PY{l+s}{y\PYZhy{}intercept: ln(z) = }\PY{l+s+si}{\PYZpc{}f}\PY{l+s}{\PYZdq{}} \PY{o}{\PYZpc{}} \PY{n}{parms}\PY{p}{[}\PY{l+s}{\PYZsq{}}\PY{l+s}{b}\PY{l+s}{\PYZsq{}}\PY{p}{]}\PY{p}{)}
         
         \PY{c}{\PYZsh{} Make the plot:}
         \PY{n}{plt}\PY{o}{.}\PY{n}{rc}\PY{p}{(}\PY{l+s}{\PYZsq{}}\PY{l+s}{text}\PY{l+s}{\PYZsq{}}\PY{p}{,} \PY{n}{usetex}\PY{o}{=}\PY{n+nb+bp}{True}\PY{p}{)}
         \PY{n}{fig}\PY{p}{,} \PY{n}{ax} \PY{o}{=} \PY{n}{plt}\PY{o}{.}\PY{n}{subplots}\PY{p}{(}\PY{l+m+mi}{1}\PY{p}{)}
         \PY{n}{ax}\PY{o}{.}\PY{n}{plot}\PY{p}{(}\PY{p}{[}\PY{n}{u1}\PY{p}{,} \PY{n}{u2}\PY{p}{]}\PY{p}{,} \PY{p}{[}\PY{n}{np}\PY{o}{.}\PY{n}{log}\PY{p}{(}\PY{n}{z1}\PY{p}{)}\PY{p}{,} \PY{n}{np}\PY{o}{.}\PY{n}{log}\PY{p}{(}\PY{n}{z2}\PY{p}{)}\PY{p}{]}\PY{p}{,} \PY{l+s}{\PYZdq{}}\PY{l+s}{co\PYZhy{}}\PY{l+s}{\PYZdq{}}\PY{p}{)}
         \PY{c}{\PYZsh{}ax.set\PYZus{}title(\PYZdq{}\PYZdl{}ln(z) vs. \PYZbs{}bar\PYZob{}u\PYZcb{}\PYZdq{})}
         \PY{n}{ax}\PY{o}{.}\PY{n}{set\PYZus{}xlabel}\PY{p}{(}\PY{l+s}{\PYZsq{}}\PY{l+s}{Averge wind speed \PYZdl{}(m}\PY{l+s}{\PYZbs{}}\PY{l+s}{ s\PYZca{}\PYZob{}\PYZhy{}1\PYZcb{})\PYZdl{}}\PY{l+s}{\PYZsq{}}\PY{p}{)}
         \PY{n}{ax}\PY{o}{.}\PY{n}{set\PYZus{}ylabel}\PY{p}{(}\PY{l+s}{\PYZdq{}}\PY{l+s}{\PYZdl{}ln(z)\PYZdl{}}\PY{l+s}{\PYZdq{}}\PY{p}{)}
         \PY{n}{ax}\PY{o}{.}\PY{n}{set\PYZus{}xlim}\PY{p}{(}\PY{p}{[}\PY{l+m+mi}{0}\PY{p}{,}\PY{l+m+mf}{0.63}\PY{p}{]}\PY{p}{)}
         \PY{n}{ax}\PY{o}{.}\PY{n}{set\PYZus{}ylim}\PY{p}{(}\PY{p}{[}\PY{o}{\PYZhy{}}\PY{l+m+mi}{20}\PY{p}{,} \PY{l+m+mf}{1.0}\PY{p}{]}\PY{p}{)}
         
         \PY{n}{plt}\PY{o}{.}\PY{n}{show}\PY{p}{(}\PY{p}{)}
         \PY{n}{fig}\PY{o}{.}\PY{n}{savefig}\PY{p}{(}\PY{l+s}{\PYZsq{}}\PY{l+s}{HW1.3\PYZus{}fig1.png}\PY{l+s}{\PYZsq{}}\PY{p}{)}
\end{Verbatim}

    \begin{Verbatim}[commandchars=\\\{\}]
y-intercept: ln(z) = -20.794415
    \end{Verbatim}

    \begin{center}
    \adjustimage{max size={0.9\linewidth}{0.9\paperheight}}{Brodzik_assignment1_files/Brodzik_assignment1_62_1.png}
    \end{center}
    { \hspace*{\fill} \\}
    
    Solving for the corresponding height:

    \begin{Verbatim}[commandchars=\\\{\}]
{\color{incolor}In [{\color{incolor}71}]:} \PY{k}{print}\PY{p}{(}\PY{l+s}{\PYZdq{}}\PY{l+s}{Momentum roughness length z\PYZus{}0m = }\PY{l+s+si}{\PYZpc{}e}\PY{l+s}{ meters}\PY{l+s}{\PYZdq{}} \PY{o}{\PYZpc{}} \PY{n}{np}\PY{o}{.}\PY{n}{exp}\PY{p}{(}\PY{n}{parms}\PY{p}{[}\PY{l+s}{\PYZsq{}}\PY{l+s}{b}\PY{l+s}{\PYZsq{}}\PY{p}{]}\PY{p}{)}\PY{p}{)}
\end{Verbatim}

    \begin{Verbatim}[commandchars=\\\{\}]
Momentum roughness length z\_0m = 9.313226e-10 meters
    \end{Verbatim}

    So my calculated momentum roughness length is about 10 Angstroms, which
seems impossibly small. But I am unsure as to what I've done wrong,
here.

Taking a different approach, the problem states that the roughness
length is 0.03 m. Using Dingman eq (3.28 and 3.29), p.~122, if this
roughness length is \(z_0\), then the average vegetation height
\(z_{veg}\) = 0.3 m = 30 cm , which seems reasonable for a ``natural
grassland''. And, the zero-plane displacement height, \(z_d\) is
therefore 0.7 * 0.3 = 0.21 m.

Using equation 3.30b to estimate friction velocity, \(u^*\):

\[u^* = \frac{\kappa [u(z_2) - u(z_1)]}{ln(\frac{z_2 - z_d}{z_1 - z_d})}\]

where:

\(\kappa\) = 0.4

    \begin{Verbatim}[commandchars=\\\{\}]
{\color{incolor}In [{\color{incolor}74}]:} \PY{n}{z\PYZus{}veg} \PY{o}{=} \PY{l+m+mf}{10.} \PY{o}{*} \PY{l+m+mf}{0.03}
         \PY{n}{z\PYZus{}d} \PY{o}{=} \PY{l+m+mf}{0.7} \PY{o}{*} \PY{n}{z\PYZus{}veg}
         \PY{n}{kappa} \PY{o}{=} \PY{l+m+mf}{0.4}
         \PY{n}{friction\PYZus{}u} \PY{o}{=} \PY{p}{(}\PY{n}{kappa} \PY{o}{*} \PY{p}{(}\PY{n}{u2} \PY{o}{\PYZhy{}} \PY{n}{u1}\PY{p}{)}\PY{p}{)} \PY{o}{/} \PY{n}{np}\PY{o}{.}\PY{n}{log}\PY{p}{(}\PY{p}{(}\PY{n}{z2} \PY{o}{\PYZhy{}} \PY{n}{z\PYZus{}d}\PY{p}{)} \PY{o}{/} \PY{p}{(}\PY{n}{z1} \PY{o}{\PYZhy{}} \PY{n}{z\PYZus{}d}\PY{p}{)}\PY{p}{)}
         \PY{k}{print}\PY{p}{(}\PY{l+s}{\PYZdq{}}\PY{l+s}{friction velocity u* = }\PY{l+s+si}{\PYZpc{}f}\PY{l+s}{ m/s}\PY{l+s}{\PYZdq{}} \PY{o}{\PYZpc{}} \PY{n}{friction\PYZus{}u}\PY{p}{)}
\end{Verbatim}

    \begin{Verbatim}[commandchars=\\\{\}]
friction velocity u* = 0.009781 m/s
    \end{Verbatim}

    So I estimate the friction velocity \(u^*\) = 0.0098 \(m\ s^{-1}\)

    Using Dingman eq. 3.35, the momentum flux is:

\[F_M = - \rho_a\ {u^*}^2\]

Letting

\(\rho_a\) = 1.2 \(kg\ m^{-3}\)

\(u^*\) = friction velocity \(m\ s^{-1}\)

    \begin{Verbatim}[commandchars=\\\{\}]
{\color{incolor}In [{\color{incolor}77}]:} \PY{c}{\PYZsh{} Units kg/m3 * m2/s2 = kg/m/s2  ?? flux units make sense,}
         \PY{c}{\PYZsh{}                                   kg m/s per unit area per unit time}
         \PY{n}{rho\PYZus{}a} \PY{o}{=} \PY{l+m+mf}{1.2} \PY{c}{\PYZsh{} kg/m\PYZhy{}3}
         \PY{n}{momentum\PYZus{}flux} \PY{o}{=} \PY{o}{\PYZhy{}}\PY{l+m+mi}{1} \PY{o}{*} \PY{n}{rho\PYZus{}a} \PY{o}{*} \PY{n}{friction\PYZus{}u}
         \PY{k}{print}\PY{p}{(}\PY{l+s}{\PYZdq{}}\PY{l+s}{Momentum flux is }\PY{l+s+si}{\PYZpc{}f}\PY{l+s}{ kg/m/s\PYZca{}2}\PY{l+s}{\PYZdq{}} \PY{o}{\PYZpc{}} \PY{n}{momentum\PYZus{}flux}\PY{p}{)}
\end{Verbatim}

    \begin{Verbatim}[commandchars=\\\{\}]
Momentum flux is -0.011737 kg/m/s\^{}2
    \end{Verbatim}

    To calculate latent heat flux \(Q_E\), I think I would need to use
Dingman eq 3.47, but I've gotten thoroughly confused and don't
understand what \(z_m\) is supposed to represent.

To calculate sensible heat flux \(Q_H\), I think I would need to use
Dingman eq 3.57, but again, I can't figure out what \(z_m\) represents.

    (Problems 4 and 5 not attempted.)

    \begin{Verbatim}[commandchars=\\\{\}]
{\color{incolor}In [{\color{incolor} }]:} 
\end{Verbatim}


    % Add a bibliography block to the postdoc
    
    
    
    \end{document}
